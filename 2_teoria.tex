\chapter{Część teoretyczna}\label{chap:teoria}
\iffalse
Ten rozdział powinien zawierać całą istniejącą teorię z~której autor będzie korzystał w~dalszej części pracy.
\fi
Rozdział ten, ma na celu przestawienie podstaw teoretycznych podjętej tematyki, wykorzystanych dalej w części badawczej. Szczególny nacisk położony będzie na ostatecznym określeniu formuł matematycznych, które zostaną zaimplementowane w poszczególnych aplikacjach. Niemniej jednak pojawiać się będą również obszerne opisy na podstawie zgromadzonej literatury tematu, co ma ostatecznie pozwolić na zgłębienie tematyki w stopniu pozwalającym na analizę otrzymanych wyników.

\section{Wprowadzenie do tematyki sieci neuronowych} \label{sec:sieci}

Badania w dziedzinie sieci neuronowych zapoczątkowane zostały przez parę naukowców -- Warrena S. McCullocha oraz Waltera Pittsa, którzy to w roku 1943 wydali publikację opisująca uproszczony model matematyczny neuronu \cite{MaP43}. Przy tej okazji, podjęli próbę zrozumienia w jaki sposób ludzki mózg, używając tych prostych komórek, jest w stanie rozwiązywać tak złożone problemy. Wskazywali tutaj na specyfikę połączeń pomiędzy neuronami, ostatecznie tworzącą -- sieć neuronową.  Kolejnym istotnym krokiem na drodze rozwoju sieci neuronowych było stworzenie modelu perceptronu w roku 1957 przez Franka Rosenblatta \cite{Ros57}. Początkowo powstał jedynie w postaci algorytmu uruchamianego na komputerze IBM 704, jednak później model został przeniesiony na specjalnie do tego celu zaprojektowane urządzenie nazwane -- "Mark 1 perceptron". Maszyna ta docelowo miała przeprowadzać rozpoznawanie obrazu korzystając z matrycy 400 fotokomórek, losowo połączonych z neuronami. Dostrajanie tego urządzenia odbywało się za pomocą potencjometrów, które były z kolei ustawiane przez niewielkie silniki elektryczne. Jednak od tamtego czasu wiele się zmieniło i takie podejście nie jest już praktykowane -- sieci neuronowe wróciły do sfery oprogramowania i nie stanowią już osobnych maszyn. Zadaniem perceptronu było przeprowadzenie procesu uczenia, a następnie dokonanie rozpoznania znaków alfanumerycznych. Jednak jego działanie nie było zupełnie zadowalające. Co prawda układ był w stanie rozpoznawać znaki, ale miał znaczne problemy z bardziej złożonymi znakami. Już w tamtym okresie stwierdzono jak ważną własnością sieci neuronowej jest jej zdolność do równoległego przetwarzania informacji, zupełnie nieznanego dla ówczesnych komputerów. Za istotną uznano również możliwość rozwiązywania nowych, aczkolwiek podobnych, problemów bez konieczności modyfikacji kodu implementacji. Nie bez znaczenia pozostawał też fakt, że w przypadku uszkodzenia części elementów, sieć była w stanie działać poprawnie. Również znany w kręgach prekursorów informatyki -- John von Neumann -- miał wpływ na kierunek rozwoju sieci neuronowych. Jako jeden z twórców architektury przetwarzania sekwencyjnego, zaproponował w roku 1966 model obliczeniowy nazwany "cellular automata" \cite{NaB66}. Główną nowością w tym modelu była możliwość współbieżnego procesowania danych, co miało znacząco wpłynąć na wydajność przeprowadzanych operacji. Jednak przed twórcami stawały w związku z tym liczne wyzwania, takie jak komunikacja i synchronizacja pomiędzy wykorzystywanymi elementami modelu. Kolejne wątpliwości zostały wskazane w publikacji wydanej w roku 1969 przez Marvina Minsky'ego oraz Seymoura Paperta \cite{MaP69}. Autorzy przedstawili problem operatora alternatywy wykluczającej (XOR), który przy użyciu ówczesnych sieci jednowarstwowych nie mógł być poprawnie rozwiązany. Pojawiło się z innych strony jeszcze kilka innych ograniczeń i pesymistycznych przewidywań, co odbiło się sporym echem wśród badaczy sieci neuronowych. Ponadto wywołało to zastój w dalszych badaniach na okres około 10 lat. Niemniej jednak, dowody na mylność założeń autorów w końcu ujrzały światło dzienne i nadszedł okres kolejnych odkryć w tej dziedzinie. W późniejszym okresie rozwój dziedziny doprowadził do powstania olbrzymiej ilości publikacji, których analiza nie jest tak niezbędna jak wskazanie wyżej opisanych prekursorów z dziedziny sieci neuronowych. Osoby zainteresowane szerszym opisem historii sieci neuronowych mogą się odnieść do książki R. Tadeusiewicza \cite{Tad93}.

\subsection{Podstawy biologiczne}

\subsection{Architektury sieci}
\subsubsection{Sieci jednokierunkowe}
\subsubsection{Sieci liniowe}
\subsubsection{Sieci nieliniowe}
\subsubsection{Sieci samoorganizujące się} 

\subsection{Algorytmy nauki}
\subsubsection{Algorytmy nauki perceptronu}
\paragraph{Propagacja wsteczna}
\paragraph{Algorytmy gradientowe}
\subsubsection{Algorytmy nauki sieci samoorganizującej się}
\paragraph{Algorytm Kohonena}
\paragraph{Algorytm gazu neuronowego}


\section{Wykorzystywane operacje graficzne}\label{sec:grafika}