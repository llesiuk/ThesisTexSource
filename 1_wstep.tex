\chapter{Wstęp}
\iffalse
Praca MUSI stanowić samodzielne opracowanie przez dyplomanta WYBRANEGO TEMATU BADAWCZEGO pod kierunkiem promotora. Temat i~zakres pracy powinien wiązać się ze specjalnością, na której studiuje dyplomant. Orientacyjna objętość pracy inżynierskiej / licencjackiej (I-go stopnia) to 50-80 stron, zaś pracy magisterskiej (II-go stopnia) -- 70-120 stron\footnote{Przy tworzeniu niniejszego szablonu wykorzystano fragmenty \cite{Nie10}}.

Wstęp rozprawy powinien jasno określać tematykę i~zakres podejmowanego problemu, np: \textit{Niniejsza praca dotyczy (inżynierii oprogramowania / sieci komputerowych / grafiki komputerowej / sztucznej inteligencji / algorytmów ewolucyjnych / technologii baz danych)... } Należy wskazać dlaczego dana tematyka została podjęta. Czy rozwiązania istniejące w~danej dziedzinie nie są wystarczające? Czy problem można rozwiązać inaczej? Czy podejmowany problem jest aktywnym tematem badawczym? Przed jakimi wyzwaniami stoi osoba podejmująca tematykę? Na tym etapie należy jedynie zarysować problem w~sposób ogólny -- na szczegółowe opisy będzie miejsce dalej.
\fi
Temat mojej pracy obejmować będzie dziedziny nauki takie jak inżynieria oprogramowania, przetwarzanie obrazu oraz sztuczna inteligencja, ze szczególnym naciskiem na tytularne sieci neuronowe. W przypadku inżynierii oprogramowania przedstawiony będzie proces powstawania kilku aplikacji pobocznych oraz finalnej w~jednej z~technologii dla urządzeń mobilnych, która ma być wykorzystana do rozpoznawania znaków języka japońskiego wprowadzonych przez użytkownika. Elementy przetwarzania obrazu pojawią się głównie w~przypadku obróbki danych w~postaci graficznej, potrzebnych w~dalszych krokach do przeprowadzenia poprawnego rozpoznania. Natomiast sztuczna inteligencja będzie szczególnie zauważalna w~części badawczej, gdyż wszelkie wyniki będą zależeć od poprawności i~rodzaju wykorzystanych algorytmów typowych dla tejże dziedziny nauki.

Wybór tematyki został podyktowany głównie rosnącym zainteresowaniem dla wykorzystania sztucznej inteligencji w~różnorodnych procesach rozpoznawania, gdzie użycie zwykłych algorytmów spoza tej specyficznej dziedziny nauki nie jest już wystarczające, gdyż daje niedostateczne wyniki. Natomiast badania w~dziedzinie sztucznej inteligencji przynoszą ciągle nowe i coraz bardziej obiecujące rezultaty, niejednokrotnie z~możliwością ich zastosowania w~rozwiązaniach końcowych. Z~drugiej strony w~ostatnim czasie znacznie rozszerzyły się możliwości wykorzystania złożonych algorytmów obliczeniowych na wcześniej wspomnianych urządzeniach mobilnych. Ich moc obliczeniowa szczególnie w~charakterze operacji czysto matematycznych przeprowadzanych na jednostkach obliczeniowych (ang. Central Processing Unit, CPU) uległa znacznemu wzrostowi i~wielu przypadkach niemal dorównuje mniej zaawansowanym komputerom.
Obecnie znane są już aplikacje służące do rozpoznawania znaków, również tych języka japońskiego, jednak brakuje zastosowania w~tym przypadku proponowanych przeze mnie sieci neuronowych. Idea ich wykorzystania skupia się na osiągnięciu możliwie najwyższego stopnia rozpoznania przy jednoczesnym umożliwieniu tolerancji na drobne błędy występujące przy wprowadzaniu znaku przez użytkownika. Osiągniecie tego celu będzie głównym priorytetem w~części badawczej, gdzie zostaną podjęte próby zastosowania różnych algorytmów nauki sieci neuronowych oraz ich parametryzacja, która jest zwykle kluczowym krokiem na drodze do uzyskania zadowalających wyników. 


\section{Cele pracy}\label{sec:cele_pracy}
Na podstawie przeglądu tematu stawiam następujące cele pracy:

\begin{itemize}
 \item Stworzenie zestawu aplikacji, począwszy od pomagającej w~zbieraniu danych, poprzez przeprowadzającą proces uczenia aż do aplikacji finalnej gdzie dochodzi do rozpoznania znaków wprowadzanych przez użytkownika
 \item Wykazanie skuteczności poszczególnych architektur oraz wybranych algorytmów nauki sieci neuronowych
 \item Wykorzystanie technologii mobilnych przy akwizycji danych oraz podczas ostatecznego procesu rozpoznawania
\end{itemize}


\section{Uzasadnienie wyboru tematu oraz przegląd literatury}
\iffalse
W tym podrozdziale należy szczegółowo uzasadnić dlaczego wybrany został taki a~nie inny temat pracy. Trzeba przede wszystkim zaprezentować aktualny stan wiedzy w~danej dziedzinie. Oznacza to konieczność omówienia książek (ew. artykułów naukowych bądź dokumentacji technicznej) z~których będzie się korzystać w~trakcie rozprawy. Następnie należy wskazać -- tym razem już konkretnie -- co nowego zamierza się zrobić. Podstawowymi celami tego podrozdziału jest wprowadzenie czytelnika w~aktualny stand danej dziedziny i~przekonanie go, że naprawdę warto zajmować się podjętym tematem.

{\color{red} Tutaj niestety będę potrzebował więcej czasu aby odświeżyć znajomość literatury, gdyż przez ten okres czasu przestałem być na bieżąco.}
\fi

Wykorzystanie sieci neuronowych w zastosowaniach, innych aniżeli komercyjne, staje się coraz popularniejsze za sprawą szerokiego dostępu do coraz szybszych jednostek obliczeniowych. Trend ten dotyczy również urządzeń mobilnych. Właściwie, w tym przypadku, wzrost mocy obliczeniowej na przestrzeni ostatnich lat jest bardziej znaczący niż ma to miejsce dla komputerów. W związku z powyższym, pojawiają się nowe możliwości zastosowań w dość specyficznym obszarze aplikacji dla urządzeń mobilnych -- głównie w dziedzinach wymagających znacznych nakładów czysto obliczeniowych, do jakich niewątpliwie należy sztuczna inteligencja, z sieciami neuronowymi na czele. Celem niniejszej pracy będzie wykorzystanie potencjału sieci neuronowych do przeprowadzenie rozpoznawania znaków języka japońskiego i udostępnienie tego rozwiązania użytkownikom urządzeń mobilnych. Ponadto, przeprowadzone zostaną rozległe badania mające na celu ustalenie jak najbardziej optymalnych parametrów działania takiej sieci neuronowej. Dalsze szczegóły dotyczące rozwiązania pojawią się w kolejnych częściach pracy. Jednak na wstępie postaram się przedstawić podstawowy schemat procesu, bez wdawania się w szczegóły wymagające nieraz rozległego podłoża teoretycznego. Zatem, pierwszym krokiem w wypracowanym przeze mnie rozwiązaniu będzie stworzenie aplikacji na urządzenia mobilne, której zadanie będzie polegało na wsparciu procesu akwizycji danych, których znaczna ilość jest niezbędna do skutecznego wykorzystania sieci neuronowych. Pobranie danych będzie polegało na wykorzystaniu dotykowych ekranów tych urządzeń, gdzie sczytane zostaną ruchy rysujące wybrane znaki. Dane te zostaną wykorzystane jako podstawa do kolejnej aplikacji, która wykorzysta wybrane algorytmy dla sieci neuronowych i ich liczne, zmienne parametry do otrzymania wyników prowadzących do finalnego rozpoznania znaków języka japońskiego. Ta ostatnia część procesu zostanie również zaprojektowana na urządzenia mobilne, co w połączeniu z łatwym sposobem wprowadzania znaków, umożliwi szybki i przystępny sposób przeprowadzenia rozpoznania.

Niemniej jednak w celu osiągnięcia powyższych założeń niezbędne będzie odniesienie się do licznych źródeł, zarówno elektronicznych, w postaci dokumentacji technicznych, jak i tych bardziej tradycyjnych -- książek, publikacji naukowych oraz artykułów. W procesie powstawania aplikacji implementującej sieci neuronowe wykorzystania zostanie technologia Java, wraz z jej niemal bezkresną dokumentacją techniczną \cite{java}. Podstawowym źródłem wiedzy w przypadku tworzenia oprogramowania na urządzenia mobilne będzie oficjalna witryna dla developerów środowiska Android \cite{android}. Wybór tej technologii został podyktowany jej powszechnym użyciem oraz dostępem do licznych narzędzi programistycznych w postaci: bibliotek, debuggerów i emulatorów urządzeń. Ponadto wyróżnić można rozległą dokumentację techniczną znacznie ułatwiająca proces powstawania aplikacji. Nie bez znaczenia pozostaje też wsparcie sprzętowe dla bardzo zróżnicowanej grupy urządzeń, co pozwoli zaoszczędzić znaczną ilość czasu i skupić się na kluczowych elementach tej pracy. Kolejne pozycje wykorzystane przeze mnie skupiają się na aspektach znacznie bardziej naukowych. Pierwszą z nich jest publikacja Prof. Ryszarda Tadeusiewicza "Sieci neuronowe" \cite{Tad93}. Jest ona świetnym wprowadzeniem do tematyki sieci neuronowych, gdyż przedstawiona została tutaj podstawowa wiedza niezbędna do dalszego zgłębiania licznych, powiązanych zagadnień. Zawarte zostały również dzieje badań w tej materii poparte uwarunkowaniami biologicznymi ułatwiające zrozumienie intencji tworzenia sieci neuronowych. Szeroko opisane zostały również kluczowe aspekty matematyczne, poparte przystępnymi przykładami. W dalszej części pojawiają się również opisy konkretnych zastosowań, gdzie omówieniu poddawane są -- kluczowe dla części badawczej tej pracy -- parametry charakterystyczne dla poszczególnych algorytmów. Kolejną pozycją jest podręcznik "Sieci neuronowe do przetwarzania informacji" autorstwa Prof. Stanisława Osowskiego \cite{Oso06}. Tutaj również na wstępie przedstawione zostają podstawy biologiczne na przykładzie pierwszych modeli sieci neuronowych oraz dalsze, obecnie znane zastosowania. Szczegółowemu opisowi podlegają modele neuronów oraz specyficzne dla nich metody uczenia. W dalszej części prezentowane są już różnego rodzaju struktury sieci przeznaczone dla innych, specyficznych zastosowań wraz z odpowiadającymi im eksperymentami numerycznymi.
W swojej pracy chciałbym się również odwołać do publikacji autorów zagranicznych -- utrwalonych w języku angielskim. W tym przypadku odwołania dotyczyć będą głównie detali opracowywanych algorytmów i wykorzystywanych metod optymalizacyjnych. Jako pierwszą postaram się przybliżyć książkę pod tytułem "Image processing and pattern recognition. Fundamentals and techniques" (Przetwarzanie obrazów i rozpoznawanie wzorców. Podstawowe zasady oraz techniki), której autorem jest Prof. Frank Y. Shih \cite{Shi10}. Prezentowane są tam jasne tłumaczenia podstawowych zasad w obranej tematyce oraz ich dalsze zastosowania w obecnie wykorzystywanych aplikacjach. Wyjaśnia kluczowe kwestie w dziedzinie przetwarzania obrazów oraz rozpoznawania wzorców, nie tylko ułatwiając proces implementacji omawianych rozwiązań, ale również zachęca do podejmowania własnych badań nad omawianymi tematami. Kluczowymi elementami, które zostaną przeze mnie wykorzystane są informacje na temat segmentacji obrazów (Część I, Rozdział 5) oraz rozpoznawania wzorców w kontekście użycia sieci neuronowych (Część I, Rozdział 9). Ponadto, zwrócę uwagę na praktyczne zastosowania klasyfikacji obrazów w przypadku rozpoznawania znaków (Część II, Rozdział 11). Kolejnym tytułem jest "Neural networks as cybernetic systems" (Sieci neuronowe jako systemy cybernetyczne) autorstwa Prof. Holka Cruse'a \cite{Cru06}. Jest to publikacja z zakresu cybernetyki biologicznej, więc nieco różni się swoją zawartości od poprzednio wymienionych. Jak również we wstępie, sam autor, zwraca uwagę na nieco inne podejście do tematyki sieci neuronowych. Mianowicie, stara się unikać przeładowania treści publikacji wzorami matematycznymi, zamiast tego starając się ilustrować omawiane treści. Co nie zmienia ostatecznego zamiaru -- umożliwienia czytelnikowi wykorzystania sieci neuronowych w zastosowaniach praktycznych. Zmienia się jedynie sposób prezentacji, nieraz ułatwiający zrozumienie omawianych problemów. Następnym źródłem do którego chciałbym się odnieść w mojej pracy jest wspólna publikacja dwóch profesorów -- Bena Kröse'a oraz Patricka van der Smagta pod tytułem "An introduction to Neural Networks" (Wprowadzenie do sieci neuronowych) \cite{KaS96}. Tym razem również, na wstępie mamy do czynienia z odniesieniami do historii dziedziny, która to analiza pozwala nie tylko na zrozumienie podstaw na początkowo prostych problemach, ale także dostrzec ogólny kierunek dalszego rozwoju. W kontekście mojej pracy istotne mogą być szeroko opisane tutaj własności sieci, takie jak: zdolność do adaptacji, generalizacji czy klastrowania przetwarzanych danych. Pojawia się też próba wskazania zbieżności z modelami biologicznymi, jednak dotychczasowe rozwiązania wciąż są uznawane za w dużym stopniu uproszczone. Stąd też autorzy ostatecznie uznają sieci neuronowe jak odseparowany i alternatywny model obliczeniowy. Kolejną książką do której chciałbym się odwołać jest "Neural Networks. A systematic introduction" (Sieci neuronowe. Systematyczne wprowadzenie), autorstwa Prof. Raúla Rojasa \cite{Roj96}. Jest ona w dużym stopniu niejako zbiorem treści prezentowanych na wykładach Uniwersytetów w Berlinie oraz Halle. Szczególny nacisk został położony na proces systematycznego rozwoju teorii sieci neuronowych, ukazany za pomocą licznych przykładów. Szczególnej uwadze zamierzam poddać fragmenty dotyczące geometrycznej interpretacji nauki perceptronu, algorytmu propagacji wstecznej (zwięźle przedstawione niejednokrotnie w sposób graficzny) oraz innych możliwych do zastosowania algorytmów. Ostatnią publikacją, którą mam na uwadze podczas tworzenia tej pracy, jest manuskrypt autorstwa Prof. Davida Kriesela pod tytułem "A brief introduction to Neural Networks" (Krótkie wprowadzenie do sieci neuronowych) \cite{Kri07}. Również w tym przypadku wstępna wersja opierała się na zbiorze wykładów wygłaszanych na jednej z niemieckich uczelni -- tym razem Uniwersytetu w Bonn. Sam autor wskazuje na podjęcie próby przedstawienia zagadnień typowych dla sieci neuronowych, w sposób przystępny także dla osób stawiający swoje pierwsze kroki w tej tematyce. Dla mnie szczególnie istotne będą fragmenty dotyczące samego procesu nauki, tworzenia i zarządzania zbiorami danych czy również parametryzacji -- podstawy do otrzymania satysfakcjonujących wyników.

\section{Układ pracy}
\iffalse
{\color{red} Tutaj należy zamieścić opis dalszej zawartości pracy.}

Struktura dalszej części szablonu jest następująca: Rozdział \ref{chap:teoria} zawiera opis teorii ... Rozdział \ref{chap:nowa_teoria} przedstawia nową teorię wprowadzoną przez autora pracy ... Rozdział \ref{chap:narzedzia} opisuje technologie i~narzędzia wykorzystane w~pracy ...  Rozdział \ref{chap:badania} przedstawia wyniki badań / opis stworzonej aplikacji ... Rozdział \ref{chap:podsumowanie} podsumowuje uzyskane wyniki oraz płynące z~nich wnioski ... W~Dodatku \ref{app:edycja} zawarto uwagi dotyczące formatowania pracy z~użyciem systemu \LaTeX. Dodatek \ref{app:plyta} zawiera płytę CD z~aplikacją stworzoną w~ramach pracy...
\fi
W dalszej części pracy pojawią się następujące rozdziały:
\begin {itemize}
\item \textbf{Rozdział \ref{chap:teoria}}: Część teoretyczna\\
Zawiera wprowadzenie teoretyczne do poruszanych w pracy tematów.
\begin {itemize}
\item \textbf{Podrozdział \ref{sec:sieci}}: Wprowadzenie do tematyki sieci neuronowych
\item \textbf{Podrozdział \ref{sec:grafika}}: Wykorzystywane operacje graficzne
\end{itemize}
\item \textbf{Rozdział \ref{chap:narzedzia}}: Technologie i~narzędzia\\
Opisuje technologie i~narzędzia wykorzystane w procesie tworzenia aplikacji jako podłoża dla części badawczej pracy.
\item \textbf{Rozdział \ref{chap:badania}}: Wyniki badań eksperymentalnych\\
Przedstawia opis stworzonych aplikacji oraz wyniki badań na nich opartych.
\item \textbf{Rozdział \ref{chap:podsumowanie}}: Podsumowanie i~wnioski\\
Podsumowuje uzyskane wyniki oraz płynące z~nich wnioski.
\item \textbf{Dodatek \ref{app:plyta}}: Płyta CD\\
Zawiera płytę CD z~aplikacjami stworzonymi w~ramach pracy.
\end{itemize}
